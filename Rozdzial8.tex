\chapter{Podsumowanie}
\section{Rekapitulacja celów i osiągnięć pracy}
W niniejszej pracy głównym celem było opracowanie, implementacja i kompleksowa analiza aplikacji mobilnej oraz układu pomiarowego, skoncentrowanych na monitorowaniu czasu reakcji, saturacji krwi i fali tętna. Praca miała na celu wykorzystanie zaawansowanych technologii w dziedzinie inżynierii biomedycznej w celu dostarczenia precyzyjnych narzędzi do pomiaru i analizy parametrów fizjologicznych.

Przeprowadzenie badań na trzech różnych grupach pacjentów, tj. osób nieaktywnych fizycznie, kolarzach oraz himalaistach, było kluczowym elementem projektu. Analiza czasu reakcji skupiała się na zrozumieniu wpływu różnego stopnia aktywności fizycznej na szybkość reakcji organizmu na bodźce. Jednocześnie, monitorowanie saturacji krwi i fali tętna miało na celu ocenę adaptacji układu krążenia wynikającej z obniżonego poziomu tlenu we krwi.

W ramach osiągnięć pracy:
\begin{itemize}
    \item Została stworzona aplikacja mobilna, która umożliwia precyzyjny pomiar czasu reakcji.
    \item Wykonano wygodną rękawicę wyposażoną w układ pomiarowy umożliwiający ciągłe monitorowanie pulsu oraz saturacji badanej osoby.
    \item Przeprowadzono badania na trzech grupach pacjentów, co pozwoliło na analizę wpływu aktywności fizycznej na parametry fizjologiczne.
    \item Wykorzystano wiele zaawansowanych technologii do efektywnej implementacji i zarządzania projektem.
\end{itemize}