\chapter{Wnioski}
\section{Implikacje dla dziedziny medycyny sportowej i rehabilitacji}
Implikacje dla dziedziny medycyny sportowej i rehabilitacji stanowią istotną część wyników uzyskanych w ramach przeprowadzonych badań nad parametrami fizjologicznymi w różnych grupach uczestników. Otrzymane wyniki posiadają potencjał do przyczynienia się do rozwinięcia praktyki medycznej w dziedzinie sportu i rehabilitacji.
\begin{itemize}
    \item Dostosowanie Programów Treningowych: Na podstawie uzyskanych wyników możliwe będzie dostosowanie programów treningowych w zależności od grupy docelowej. Osoby nieaktywne fizycznie, kolarze i himalaiści mogą korzystać z dedykowanych planów treningowych, uwzględniających indywidualne cechy fizjologiczne.
    \item Źródło Wiedzy dla Specjalistów: Wyniki badania staną się źródłem cennych informacji dla specjalistów z dziedziny medycyny sportowej i rehabilitacji. Publikacja uzyskanych wyników umożliwi lekarzom, fizjoterapeutom i trenerom korzystanie z aktualnej wiedzy, aby lepiej wspierać swoich pacjentów w osiąganiu zdrowia i celów sportowych.
    \item Optymalizacja Procesu Rehabilitacji: Dla dziedziny rehabilitacji, wyniki badania mogą dostarczyć informacji na temat wpływu aktywności fizycznej na proces rekonwalescencji. Identifikacja różnic pomiędzy grupami może pomóc w optymalizacji procesów rehabilitacyjnych dla osób aktywnych oraz tych, które dopiero rozpoczynają aktywność fizyczną.
    \item Nowe Technologie w Pomiarze Czasu Reakcji: Przedstawione badania otwierają również pole do wprowadzenia nowoczesnych technologii w pomiarze czasu reakcji. Wykorzystanie nowego oprogrmaowania może znacząco usprawnić dokładność i precyzję pomiarów.
\end{itemize}

\section{Wnioski z uzyskanych wyników}
Każda grupa wykazywała unikalne odpowiedzi fizjologiczne na specyficzne przygotowane środowisko, co sugeruje, że adaptacja organizmu zależy od poziomu aktywności fizycznej i doświadczenia środowiskowego. Wyniki te mogą mieć implikacje dla treningu sportowego, medycyny i dostosowania się organizmu do różnych warunków środowiskowych. Dalsze badania z większą liczbą uczestników są zalecane, aby bardziej szczegółowo zrozumieć te różnice i potwierdzić obserwowane tendencje. 
\section{Propozycje dalszych badań}
Badania naukowe to dynamiczny proces, który wymaga ciągłego rozwijania i eksplorowania nowych obszarów. Przedstawiono sugestie dotyczące kontynuacji i pogłębienia analiz związanych z omawianą tematyką, w celu wyznaczenie kierunków, w jakich można rozwijać badania oraz zgłębianie tych zagadnień.

\subsection{Rozszerzenie próby badawczej}
Rozszerzenie próby badawczej o większą liczbę uczestników stanowiłoby kluczowy krok w kierunku uzyskania bardziej reprezentatywnych i pewnych wyników. Obecna próba obejmująca osoby nieaktywne fizycznie, kolarzy i himalaistów stanowi solidne podstawy, ale zwiększenie liczby uczestników pozwoliłoby na bardziej precyzyjne analizy porównawcze oraz lepsze zrozumienie różnic między poszczególnymi grupami.

Zwiększona liczba uczestników umożliwiłaby również dokładniejszą analizę zróżnicowania w obrębie każdej grupy, co jest istotne ze względu na indywidualne cechy fizjologiczne i poziom aktywności. Takie podejście przyczyniłoby się do bardziej wiarygodnych wniosków dotyczących wpływu aktywności fizycznej na parametry fizjologiczne.

Dodatkowo, rozszerzenie próby badawczej stanowiłoby solidne podstawy do przeprowadzenia analizy statystycznej, co wpłynęłoby na siłę dowodową uzyskanych wyników. Otrzymane dane mogą posłużyć do identyfikacji ewentualnych trendów oraz wykrycia subtelnych różnic między poszczególnymi grupami uczestników.
\subsection{Badania w środowisku hipoksyjnym}
Środowisko hipoksyjne to warunki atmosferyczne, w których poziom tlenu jest znacznie niższy niż w warunkach normoksji \cite{hipoksja}, czyli standardowego poziomu tlenu na poziomie morza. Hipoksja może występować naturalnie w wysokich górach, na dużych głębokościach pod wodą, a także sztucznie, np. w komorach hipoksycznych, które są wykorzystywane do symulacji warunków wysokogórskich.

Celem badania byłoby Badanie wpływu hipoksemii na parametry zdrowotne podczas ćwiczeń na rowerku stacjonarnym, w celu określenia czy niski poziom tlenu wpływa na parametry zdrowotne w trakcie aktywności fizycznej. Wyniki mogą dostarczyć informacji na temat adaptacji organizmu do warunków hipoksji i ewentualnych korzyści zdrowotnych lub ryzyka.