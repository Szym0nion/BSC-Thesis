\chapter{Wprowadzanie}
%%%%%%%%%%%%%%%%%%%%%%%%%%%%%%%%%%%%%%%%%%%%%%%%%%%%%%%%%%%%%%%%%
%%%%%%%%%%%%%%%%%%%%%%%%%%%%%%%%%%%%%%%%%%%%%%%%%%%%%%%%%%%%%%%%%
\section{Kontekst badawczy}
Współczesne postępy w dziedzinie inżynierii biomedycznej umożliwiają rozwijanie innowacyjnych rozwiązań monitorowania parametrów fizjologicznych w czasie rzeczywistym. W ramach tego kontekstu, praca inżynierska skupia się na projektowaniu i implementacji aplikacji mobilnej oraz układu pomiarowego, mających na celu kompleksową analizę czasu reakcji, saturacji krwi i fali tętna.

Stan wiedzy na temat monitorowania fizjologicznych wskaźników zdrowia wskazuje na istnienie luk w dotychczasowych badaniach, zwłaszcza w kontekście różnic indywidualnych pomiędzy grupami pacjentów o zróżnicowanym stopniu aktywności fizycznej. Zidentyfikowane luki obejmują brak precyzyjnych narzędzi do pomiaru czasu reakcji oraz potrzebę zrozumienia adaptacji układu krążenia w zależności od rodzaju wykonywanej aktywności fizycznej, szczególnie w tak specyficznych warunkach z jakimi mają doczynienia zawodowi himalaiści.
%%%%%%%%%%%%%%%%%%%%%%%%%%%%%%%%%%%%%%%%%%%%%%%%%%%%%%%%%%%%%%%%%
%%%%%%%%%%%%%%%%%%%%%%%%%%%%%%%%%%%%%%%%%%%%%%%%%%%%%%%%%%%%%%%%%
\section{Cel i zakres pracy}
Głównym celem niniejszej pracy jest opracowanie, implementacja oraz kompleksowa analiza aplikacji mobilnej oraz układu pomiarowego, mających na celu monitorowanie czasu reakcji, saturacji krwi i fali tętna. Praca skupia się na zastosowaniu zaawansowanych technologii w dziedzinie inżynierii biomedycznej, aby dostarczyć precyzyjne narzędzia do pomiaru i analizy parametrów fizjologicznych.

Dodatkowo zamiarem jest przeprowadzenie badań na trzech różnych grupach pacjentów: osób nieaktywnych fizycznie, kolarzach oraz himalaistach. Analiza czasu reakcji ma na celu zrozumienie wpływu różnego stopnia aktywności fizycznej na szybkość reakcji organizmu na bodźce. Jednocześnie, monitorowanie saturacji krwi i fali tętna ma na celu ocenę adaptacji układu krążenia wynikającą z obniżonego poziomu tlenu we krwi.
%%%%%%%%%%%%%%%%%%%%%%%%%%%%%%%%%%%%%%%%%%%%%%%%%%%%%%%%%%%%%%%%%
%%%%%%%%%%%%%%%%%%%%%%%%%%%%%%%%%%%%%%%%%%%%%%%%%%%%%%%%%%%%%%%%%
\section{Motywacja do przeprowadzenia badań}
Realizacja tych badań stanowi odpowiedź na aktualne wyzwania związane z monitorowaniem parametrów fizjologicznych, jak również przygotowaniem himalaistów do podróży. Motywacją do przeprowadzenia badań w ramach tego projektu jest kilka kluczowych aspektów:
\begin{itemize}
    \item Zastosowanie Zaawansowanych Technologii: Projekt jest motywowany pragnieniem wykorzystania różnorodnych, nowoczesnych technologii w celu stworzenia efektywnego narzędzia do monitorowania parametrów zdrowotnych oraz umożliwienia prowadzenia badań. Wykorzystano bogactwo dostępnych technologii w celu dostarczenia kompleksowego i zaawansowanego rozwiązania, które pozwoli użytkownikowi na monitorowanie kluczowych wskaźników zdrowotnych.
    \item Wszechstronna Analiza Wpływu Aktywności Fizycznej: Badania skierowane na trzy zróżnicowane grupy pacjentów (osoby nieaktywne fizycznie, kolarze, himalaści) mają na celu zrozumienie, jak różne poziomy aktywności fizycznej wpływają na badane parametry. Motywacja tkwi w potrzebie wszechstronnej analizy reakcji organizmu na aktywność fizyczną przy obniżonym poziomie tlenu, co ma potencjalne znaczenie dla dostosowania programów zdrowotnych.
    \item Wprowadzenie Nowych Współczynników Analizy Zdrowia: Zaproponowano wprowadzenie nowych mierników do oceny zdrowia, takich jak czas reakcji, saturacja krwi i fala tętna. Motywacją jest poszukiwanie bardziej precyzyjnych i spersonalizowanych wskaźników, które mogą dostarczyć istotnych informacji na temat zdrowia pacjentów oraz przygotowania ich do specjalistycznego wysiłku.
    \item Zastosowanie W Medycynie Sportowej i Rehabilitacji: Wyniki badań mogą mieć istotne znaczenie dla dziedzin medycyny sportowej i rehabilitacji, sugerując dostosowanie programów zdrowotnych do indywidualnych potrzeb pacjentów, w szczególności dla poprawy wydolności himalaistów. Motywacją jest poszukiwanie praktycznych zastosowań uzyskanej wiedzy w obszarach, gdzie precyzyjne monitorowanie parametrów fizjologicznych jest kluczowe.
    \item Współpraca z Różnymi Grupami Pacjentów: Motywacją jest chęć dostarczenia wartościowej wiedzy dla różnych grup pacjentów, zarówno tych aktywnych fizycznie, jak i tych, którzy prowadzą siedzący tryb życia. Wsparcie zdrowia zarówno dla sportowców, jak i osób nieaktywnych fizycznie jest istotnym celem tej pracy.
\end{itemize}

