% Ustawienie wymaganej dokumentami SZJK czcionki 
\usepackage[sfdefault]{carlito}

% \usepackage{fontspec}
% \setmainfont{Barlow}

% Dołączenie pakietów wprowadzających częściowe spolszczenie
\usepackage{polski}

% Ustawienie interlinii
\linespread{1.5}

% Ustawienie metadanych w pliku pdf
\author{\Autor}
\title{\TematPracy}
\date{\Rok}
\AtBeginDocument{
    \hypersetup{
        pdftitle=\TematPracy,
        pdfauthor=\Autor,
        pdfsubject=\TematPracyANG,
        pdfkeywords=\SlowaKluczowe}
}

% Ustawienie kropek po numerach sekcji (różnych poziomów)
% UWAGA - nie dodaje kropek po nr sekcji w spisie treści
\usepackage{secdot}
\sectiondot{section}
\sectiondot{subsection}
\sectiondot{subsubsection}

% Dołączenie pakietu umożliwiającego manipulację tekstem małe/wielkie litery nazwy własne itp.
\usepackage{mfirstuc}
\usepackage{xstring}

% Dołącz i skonfiguruj pakiet zapewniający aktywne i kolorowe hiperłącza
\usepackage[colorlinks]{hyperref}

% Dołącz pakiet umożliwiający generację zakładek
\usepackage{bookmark}

% Deklaracja nowego licznika o nazwie EndOfTOC (będzie przechowywał nr strony, na której kończą się spisy treści, rysunków...)
\newcounter{EndOfTOC}

\usepackage{geometry}
\geometry{
	paper = a4paper,
	top = 2.5cm,
	bottom = 2.5cm,
	inner = 2.5cm,
	outer = 2.5cm,
	bindingoffset = 1cm,
	headheight = 1.25cm,
	footskip = 1.35cm,
% 	showframe,
}

% Dołączenie pakietu umożliwiającego definiowanie stylów stron 
\usepackage{fancyhdr}
% Definicje stylów stron 
\fancypagestyle{SpisPierwsza}{
	\renewcommand{\headrulewidth}{0pt}
	\renewcommand{\footrulewidth}{1pt}
	\fancyhead[RO,RE]{}
	\fancyhead[LO,LE]{}
	\fancyhead[CO,CE]{}
	\fancyfoot[CO,CE]{\thepage}
}
\fancypagestyle{Spis}{
	\renewcommand{\headrulewidth}{1pt}
	\renewcommand{\footrulewidth}{1pt}
	\fancyhead[RO,RE]{}
	\fancyhead[LO,LE]{}
	\fancyhead[CO,CE]{\nouppercase{\rightmark}}
	\fancyfoot[CO,CE]{\thepage}
}
\fancypagestyle{plain}{
	\renewcommand{\headrulewidth}{0pt}
	\renewcommand{\footrulewidth}{1pt}
	\fancyhead[RO]{}
	\fancyhead[LE]{}
	\fancyhead[RE]{}
	\fancyhead[LO]{}
	\fancyhead[CO,CE]{}
	\fancyfoot[RO]{}
	\fancyfoot[LE]{}
	\fancyfoot[CO,CE]{\thepage}
	\fancyfoot[RE,LO]{}
}
\fancypagestyle{Praca}{
	\renewcommand{\headrulewidth}{1pt}
	\renewcommand{\footrulewidth}{1pt}
	\fancyhead[RO]{\nouppercase{\rightmark}}
	\fancyhead[LE]{\nouppercase{\leftmark}}
	\fancyhead[RE]{}
	\fancyhead[LO]{}
	\fancyhead[CO,CE]{}
	\fancyfoot[RO]{}
	\fancyfoot[LE]{}
	\fancyfoot[CO,CE]{\thepage}
	\fancyfoot[RE,LO]{}	
}

% Dołączenie pakietu umożliwiającego manipulację spisem treści
\usepackage{tocloft}
% Ustaw spis treści tak, by po numerze rozdziału i sekcjach różnego poziomu w spisie treści była kropka
\renewcommand{\cftchapaftersnum}{.}%
\renewcommand{\cftsecaftersnum}{.}%
\renewcommand{\cftsubsecaftersnum}{.}%
\renewcommand{\cftsubsubsecaftersnum}{.}%
\renewcommand{\cftparaaftersnum}{.}%
\renewcommand{\cftsubparaaftersnum}{.}%
% Ustaw spis rysunków i tabel tak, by po numerze rysunku/tabeli w spisie rys/tab była kropka
\renewcommand{\cftfigaftersnum}{.}%
\renewcommand{\cfttabaftersnum}{.}%


% Pakiety i ustawienia dotyczące grafik
\usepackage{graphicx}
\graphicspath{{./Ilustracje/}}
\usepackage{subfig}
\usepackage{caption}
\captionsetup{labelsep=period}
\renewcommand{\figurename}{Rys.}
\renewcommand{\tablename}{Tab.}

% Skorowidz haseł
\usepackage{makeidx}
\makeindex
\usepackage{idxlayout}

% Wykaz oznaczeń
\usepackage{nomencl}
\makenomenclature

% Definicje zmiennych binarnych sterujących obecnością Skorowidza haseł, Wykazu oznaczeń, Spisów rys/tab/listingów
\newboolean{SpisRysunkow}
\newboolean{SpisTabel}
\newboolean{SpisProgramow}
\newboolean{SpisOznaczen}
\newboolean{Skorowidz}

% Definicje układu stron rozpoczynających rozdziały
\makeatletter
\def\@makechapterhead#1{%
  \vspace*{5cm}%
  {\parindent \z@ \raggedright \normalfont
    \ifnum \c@secnumdepth >\m@ne
        \huge\bfseries
        \thechapter.\ 
        \nobreakspace{}
    \fi
    \interlinepenalty\@M
    \huge
    \bfseries #1\par\nobreak
    \vskip 40\p@
  }}
\makeatother

\makeatletter
\def\@makeschapterhead#1{%
  \vspace*{5cm}%
  {\parindent\z@\raggedright\normalfont%
    \ifnum\c@secnumdepth>\m@ne%
        \huge\bfseries%
        \nobreakspace{}
    \fi
    \interlinepenalty\@M%
    \huge
    \bfseries #1\par\nobreak%
    \vskip 40\p@%
  }}
\makeatother

% Definicje poleceń
\newcommand{\DolaczRozdzial}[1]{
    \input{#1}
    \clearpage{\pagestyle{empty}\cleardoublepage}
}

\newcommand{\Bibliografia}[2]{
    \bibliographystyle{#1}
    \addcontentsline{toc}{chapter}{\bibname}
    \bibliography{#2}
    \clearpage{\pagestyle{empty}\cleardoublepage}
}

\newcommand{\Skorowidz}{
    \ifthenelse{\boolean{Skorowidz}}
    {
        \printindex
        \addcontentsline{toc}{chapter}{\indexname}
        \clearpage{\pagestyle{empty}\cleardoublepage}
    }
    {}
}

% Dołączenie pakietu, konfiguracja wypunktowań, definicja nowego konspektu listy numerowanej
\usepackage{enumitem}
\setlist{topsep=0pt,itemsep=0pt,partopsep=0pt,parsep=0pt}%
\newlist{inparaenum}{enumerate}{3}%
\setlist[inparaenum]{nosep}%
\setlist[inparaenum,1]{label=\arabic*.}%
\setlist[inparaenum,2]{label=\arabic{inparaenumi}.\arabic*.}%
\setlist[inparaenum,3]{label=\arabic{inparaenumi}.\arabic{inparaenumii}.\arabic*.}%

% Dołączenie pakietów i konfiguracja narzędzi do umieszczania kodów programów
\usepackage{fancyvrb}
\usepackage{listings}
\renewcommand{\lstlistingname}{Program}
\usepackage{listingsutf8}
% Ustaw spis programów tak, by po numerze programu (listingu) w spisie programów była kropka
\usepackage{xpatch}
\makeatletter
\xpatchcmd\lst@MakeCaption{\protect\numberline{\thelstlisting}\lst@@caption}{\protect\numberline{\thelstlisting.}\lst@@caption}{}{}
\makeatother

% Dołączenie dodatkowych pakietów wspomagających używanie bibliografii, tabel...
\usepackage{cite}
\usepackage{url}
\usepackage{array}
\usepackage{multirow}